\documentclass[11pt,a4paper]{article}
\usepackage[latin1]{inputenc}
\usepackage{amsmath}
\usepackage{amsfonts}
\usepackage{amssymb}
\usepackage{graphicx}
\usepackage{longtable}
\usepackage{tabularx}
\usepackage{enumitem}
\usepackage{url}
\usepackage[margin=0.8in]{geometry}
\usepackage[toc,page]{appendix}
\usepackage{etoolbox}
\usepackage{morefloats}
\usepackage{multirow}
\usepackage[hidelinks]{hyperref}
\usepackage{float} % Allows putting an [H] in \begin{figure} to specify the exact location of the figure
\usepackage{verbatim}
\usepackage{listings}

\graphicspath{{img/}}

\patchcmd{\thebibliography}{\section*}{\subsection}{}{}

% Table padding
\renewcommand{\arraystretch}{1.5}

\begin{document}

\begin{titlepage}

\begin{center}
\includegraphics[width=0.5\textwidth]{img/University_Logo}\\

\textsc{\LARGE Swansea University }\\[0.5cm]
\textsc{\large MEng Computing }\\[2cm]

{ \huge \bfseries Group Project CS-M04}\\[0.2cm]
\textsc{\large White Rock Digital Trails: Interim Report}\\[1.5cm]

\begin{minipage}{0.4\textwidth}
\begin{flushleft}

\emph{Authors:}\\
Adam \textsc{Barrell} {\scriptsize \emph{(632975)}} \\
Thomas \textsc{Milner} {\scriptsize \emph{(637755)}} \\
Lewis \textsc{Hancock} {\scriptsize \emph{(646113)}} \\
Christopher \textsc{Lewis} {\scriptsize \emph{(636238)}} \\

\end{flushleft}
\end{minipage}
\begin{minipage}{0.4\textwidth}
\begin{flushright}

\emph{Supervisor:}\\
Parisa \textsc{Eslambolchilar}

\end{flushright}
\end{minipage}\\[1.3cm]

{\today}
\end{center}

\end{titlepage}

\newpage 

\tableofcontents

\newpage
\section{Introduction}
%- Purpose of Document (who it's for, outline Digital Trails briefly again).
%- Outline of Document (what we cover in the document).

\section{Technology Choices}
%- What APIs, Libraries, IDEs etc we are using, compared with other options we had. Argue our choices.

\section{Project Progress}
%- THIS IS THE OVERVIEW SECTION
%- Show the subsystem designs and explain the current structure of the project. 
%- Possible to move the risk management and schedule sections up here if people feel it reads better. Depends how big those sections get.
%- Mention Amazon EC2 for testing purposes

\subsection{Database}

%- Why do we need a database, what data will it store?
A relational database has been created to store persistent data for the web portal and Android application.
The database stores data relating to entities such as walks, way points, users and media locations.
The database is currently hosted by an Amazon EC2 server (see technology choices) using MySQL.

%- How was the database designed?
The database schema was designed using Visual Studio's entity designer which allows developers visually develop databases.
This tool allows tables to be created on a canvas where relationships and attributes can be modified.
A table was created for each entity or concept that required persistent storage as shown in Figure \ref{fig:DatabaseSchema}.
Columns were then added to each table to represent the attributes of each entity.
For example, the \emph{EnglishWalkDescription} table has an \emph{Id}, \emph{Title}, \emph{ShortDescription} and \emph{LongDescription}. The column \emph{Id} is mandatory for standalone entities as the rest of their properties are identified by this unique key. 

%- How was the database be generated from this model?
The Visual Studio entity designer was able to generate an SQL script to create the database from the design shown in Figure \ref{fig:DatabaseSchema}. This script was executed on the MySQL server which in turn created the tables designed in the entity designer.

%- How will the database be exposed if not directly?
The White Rock Trails database is not publicly exposed and can only be interfaced through a public facing API (see section XXXXX). 
This will ensure the robustness of the database as additional business logic from the API protects the underlying database from erroneous data and unauthorized access.

%- What applications will consume data from this database?
The web portal and Android application will both consume data from this database through the public API. 
The web portal will consume data with every request made by users from their web browser. 
In contrast, the Android application will synchronise a local copy of the database through the API. 
This ensures that the application can be used offline as many walks will not be in range of an internet access point.

%- How does it's design conform to the requirements?
The database design conforms to the requirements required by the client which are defined in the Initial Document. 
More specifically, the design shown in Figure \ref{fig:DatabaseSchema} is based on the initial schema design provided by the client. 
The initial schema contained tables that did not have a high normal form. 
English and Welsh translations were present in the same tables for both the walk and way point descriptions. 
These tables were normalised as shown in Figure \ref{fig:DatabaseSchema} by moving the English and Welsh descriptions into separate tables.
In addition, the media locations for each way point were also present in the same table.
This was normalised by moving different media types such as images, audio and video into tables \emph{WaypointImage}, \emph{WaypointAudio} and \emph{WaypointVideo} respectively.

%- Describe the relationships (one-one) (one-many) (many-many)
Relationships were defined between tables using Visual Studio's entity designer. 
The designer supports a range of associativities such as one-one, one-many and many-many. 
Some of these associativities were used in the White Rock Trails schema as shown in Figure \ref{fig:DatabaseSchema}.
For example, the \emph{Walk} entity can have many \emph{WalkReviews}, one \emph{User} who created it, many \emph{Waypoints}, one \emph{EnglishWalkDescription} and one \emph{WelshWalkDescription}.

\begin{figure}[h!]
\centering
\includegraphics[angle=90, width=1\linewidth]{./img/DatabaseSchema}
\caption{Schema of the White Rock Trails relational database.}
\label{fig:DatabaseSchema}
\end{figure}


\subsection{User Interfaces}
%- Heuristic Evaluations on similar products
%-- Explain how we are doing the evaluation blahblah nielsen etc.
%-- List of tasks for evaluators to complete (we should really ALL be involved in doing the evaluations).
%-- Problems found when completing each task
%- Our prototype UI 1
%-- show how it meets reqs, how heuristics helped design.
%-- Heuristics on this etc.
%- Our prototype UI 2
%-- show it meets reqs, how heuristics helped design.
%-- Heuristics
%- repeat as necessary.

\subsection{Android Application}
%- ContentProvider
%- Authorisation
%- Google Maps
%- Current design to sync with server
%- Chris can chuck in his Fragments here, so he has some code to show.

\subsection{Web Portal}

\subsubsection{API}
\subsubsection{Web Application}
\subsubsection{Object Relation Mapper}

\section{Programming Guidelines}
%- I'll write android guidelines.
%- Adam/Tom get on the web guidelines.

\section{Risk Analysis Review}
%- Explain risks we dealt with, cross reference to our initial table.
%- Add new risks that arose (Poor work environment, lab computers suck, hudl broken, hudl missing USB driver - not supported as a dev. device).

\section{Project Schedule Review}
%- What sprints we have completed.
%- What reqs / specifications are completed Chris Lewis' favourite job!).
%- Are we ahead or behind schedule? (hard for us to answer)
%- Did we have to modify any requirements / specs to get this far? Did we add any or drop any?
%- Create new gantt chart and compare it with the initial one.
%- Update online sprint software, make it look like we completed sprints on time.
%- Client feedback so far, previous client meetings, planned meetings, launch event we attended etc.

\section{Summary}
%- What we have planned next, I guess.

%References as subsection
\newpage
\bibliographystyle{plain}
\bibliography{bibliography}
\end{document}